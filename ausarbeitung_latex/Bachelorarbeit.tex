\documentclass[accentcolor=tud1c,colorback,ngerman,12pt] {tudreport}
\usepackage[utf8]{inputenc}
\usepackage{babel}
\usepackage{acronym}
\usepackage{listings}
\usepackage{tabularx}
\usepackage{url}
\usepackage{hyperref}
\usepackage{multirow}

\lstset{language=C,basicstyle=\small, breaklines=true, showtabs=false, showspaces=false, showstringspaces=false}

\renewcommand{\lstlistingname}{Quelltext}
\lstset{literate=%
{Ö}{{\"O}}1
{Ä}{{\"A}}1
{Ü}{{\"U}}1
{ß}{{\ss}}1
{ü}{{\"u}}1
{ä}{{\"a}}1
{ö}{{\"o}}1
}

\setinstitutionlogo[height]{bilder/rechnersysteme-transparent}

\begin{document}
\title{Automatisches bauen und parallelisieren von Microblaze Systemen}
\subtitle{Masterthesis}
\subsubtitle{Christian Hohenbrink \hfill  \\4. Januar 2017}

\maketitle
\chapter*{Erklärung gemäß § 22 Abs. 7 APB}

Hiermit erkläre ich gemäß § 22 Abs. 7 der Allgemeinen Prüfungsbestimmungen (APB) der Technischen Universität Darmstadt in der Fassung der 4. Novelle vom 18. Juli 2012, dass ich die Arbeit selbstständig verfasst und alle genutzten Quellen angegeben habe und bestätige die Übereinstimmung von schriftlicher und elektronischer Fassung.\\ \\ \\ \\

\parbox{8cm}{\centering Darmstadt, den 4. Januar 2017\hrule
\strut \centering\footnotesize Ort, Datum} \hfill\parbox{8cm}{\phantom{Darmstadt, den 4. Januar 2017} \hrule
\strut \centering\footnotesize Name}

\vfill

\noindent \textbf{Fachbereich Elektro- und Informationstechnik}\\
Institut für Datentechnik\\
Fachgebiet Rechnersysteme\\
Prüfer: Prof. Dr.-Ing. Christian Hochberger\\
Betreuer: M.Sc. Kris Heid

\tableofcontents

\chapter{Einleitung}
\section{Einführung in die Thematik}
Am Fachgebiet Rechnersysteme der TU Darmstadt und der Professsur für eingebettete Systeme der TU Dresden wurde der CPU-Core
SpartanMC entwickelt, welcher mit einer Daten- und Instruktionsbreite von 18 Bit ideal für die Verwendung in FPGAs geeignet ist. %TODO: auf SPMC-Seite verweisen???
Ebenso wurde eine Toolchain entwickelt, die es dem Nutzer ermöglicht über die grafische Oberfläche JConfig Systeme zu beschreiben und
im Anschluss über Make-Regeln weitere Schritte wie Synthese und Simulation durchzuführen.\\
Der SpartanMC wir am Fachgebiet Rechnersysteme zur Forschung an Manycore SoC verwendet. Eine dieser Forschungsarbeiten stellt das Programm \textmu\/Streams da,
welches auf Basis des Source-to-Source Compilers Cetus entwickelt wurde, um die Leistungsfähigkeit von Manycore Systemen zu steigern. %TODO: auf CETUS-Seite verweisen???
Hierzu kann der Programmierer den Sourcecode mit Annotationen versehen, die den Code in verschieden Aufgabenblöcke unterteilt. Dabei ergibt sich
eine Pipeline-Struktur, in der die Ergebnisse einzelener Berechnungen über Core-Konnektroren weitergegeben werden.
\textmu\/Streams kann sowohl eine Hardwarebeschreibung für ein Manycore System generieren, als auch Software für die einzelenen Cores, 
welche mit der SpartanMC Toolchain konform sind.\\
\section{Ziele der Arbeit}
Um zu zeigen, dass diese Art der Perfromancesteigerung nicht nur auf den SpartanMC begrenzt ist, ist das primäre Ziel dieser Arbeit, den Xilinx Microblaze 
in die bestehende SpartanMC Toolchain zu integerieren.\\
Hierzu soll die Konfiguration des Microblaze über den SpartanMC Systembuilder JConfig ermöglicht werden. Ebenso ist es erforderlich, für einfache
I/O-Anwendungen einen UART IP-Core für den Microblaze zu integrieren. Als Pendant zu den Core-Konnektoren soll es ebenfalls möglich sein FSL-Blöcke von Xilinx
dem System hinzuzufügen, um die Kommunikation zwischen den einzelnen Prozessoren zu ermöglichen.\\
Sollte es für die Integration notwendig sein, ist die Toolchain ebenfalls anzupassen, um die Generierung von lauffähigen Microblaze Systemen zu ermöglichen.
Hierbei ist darauf zu achten, dass die Funktionalität sowohl für Systeme, die nur aus SpartanMC oder Microblaze bestehen, als auch für Systeme, die beide Prozessoren beinhalten, 
nicht negativ beeinflusst wird.\\\\
Optional wäre es wünschenswert, die erfolgreiche Paralellisierung eines Microblaze Programms durch \textmu\/Streams zu zeigen. Hierzu wäre es notwendig, 
\textmu\/Streams soweit anzupassen, dass eine entsprechende Hardwarekonfiguration mit den neu integrierten Komponenten erstellt werden kann und
die Aufrufe der Core-Konnektoren durch Aufrufe der FSL-Blöcke ersetzt werden.
\section{Gliederung der Arbeit}
Im folgenden Kapitel werden die Grundlagen für diese Arbeit beschrieben. Diese umfasst neben der Beschreibung der verwendeten Software-Tools, die Erläuterungen zum
Aufbau der SpartanMC Toolchain, sowie Erklärungen zu verwendeten Dateiformaten und Busbeschreibungen. Anschließend wird ein Konzept mit allen notwendigen Schritten 
für die Integration des Microblaze in die SpartanMC Toolchain erarbeitet. Das darauffolgende Kapitel beschreibt die Realisierung dieser Konzepte. In einem weiteren 
Abschnitt wird auf alle notwendigen Schritte eingegangen, um ein lauffähiges System zu generieren. Darauf folgend wird eine Evaluation zum Grad der erreichten Integration
durchgeführt. Abschließend folgt ein kurzer Ausblick auf mögliche Fortsetzungen der Arbeit, sowie eine kurze Zusammenfassung.
%\textit{kursiver Text} \\
%\texttt{besonderer Text} \\
%normaler Text
%\begin{itemize}
%\item[\textbf{1)}] \textbf{Aufzählung1}\\
%Text der Aufzählung
%\item [\textbf{2)}]\textbf{Aufzählung2}
%\item [\textbf{3)}]\textbf{Aufzählung3}
%\item [\textbf{4)}]\textbf{Aufzählung4}
%\item [\textbf{5)}]\textbf{Aufzählung5}
%\end{itemize}
%Quellenverlinkung \cite{UCS_XDL_UseCaseSenarios}


\chapter{Grundlagen}
\section{SpartanMC Entwicklungsumgebung}
\subsection{JConfig}
Der Systembuilder der Entwicklungsumgebung, in Java programmiert generiert er eine Toplevel verilog Beschreibung
des Systems und alle notwendigen Dateien um die Toolchain zu füttern.
\subsection{SpartanMC Toolchain}
Startet die Tools für 
\subsection{XML-Modulbeschreibung}

\section{Xilinx Embedded Development Kit}
\subsection{Microblaze}

\subsection{XPS}

\subsection{SDK}

\subsection{FSL}

\subsection{Speicherinitilaisierung im Microblaze}

\section{AMBA AXI4}

\section{VHDL-Konzept: Library}

%\section{Software Tools}

%\section{SpartanMC Hex Format}

%\section{make-toolchain}

%\begin{figure}
%\centering
%\includegraphics[height=16cm, angle=90]{./bilder/XDL}
%\caption{Die drei Haupttypen des XDL-Formats und ihre Syntax}
%\label{fig:XDL-Format}
%\end{figure}
\chapter{Konzeption und Implementation}
\section{Anforderungen}
Ziel dieser Arbeit ist es, den Xilinx Microblaze in die bestehende SpartanMC Entwicklungsumgebung zu integrieren. Um dies zu erreichen, muss zunächst der Microblaze in JConfig eingebunden werden. Der CoreGenerator von Xilinx bietet keine Alternative, da lediglich das Microblaze Micro Controller System (MCS) \cite{MCS} zur Verfügung steht. Dies ist ein System bestehend aus einem Microblaze, Speicher und grundlegenden Peripherien, wie einem UART-Core und Timern. Die Konfigurierbarkeit des Microblaze innerhalb des MCS ist stark eingeschränkt. Features wie Caches oder FSL-Interface stehen nicht zur Verfügung und die Pipeline ist immer dreistufig. Da auf die umfangreiche Konfigurierbarkeit und das FSL-Interface nicht verzichtet werden soll, muss der Microblaze in JConfig integriert werden. Um dies zu erreichen, müssen zunächst sowohl eine Hardwarebeschreibung, als auch eine XML-Modulbeschreibung für den Microblaze erstellt werden. Dies ist ebenso erforderlich für den UART IP-Core und den FSL-Bus IP-Core. Bei der Erstellung der Hardware- und der XML-Modulbeschreibung, ist darauf zu achten, dass dem Nutzer es ermöglicht wird, die Parameter der neuen Komponenten konfigurieren zu können.\\
Es ist außerdem erforderlich, dass Anpassungen an der SpartanMC Toolchain vorgenommen werden. Dies ist notwendig, um die automatische Speicherinitialisierung für den Microblaze zu ermöglichen, da die bestehende Methode der Speicherinitialisierung für den SpartanMC nicht ohne Änderungen für den Microblaze anwendbar ist. Desweiteren muss der Compiler für den Microblaze in die Toolchain integriert werden, da einige der Hardwarebeschleuniger über spezielle Instruktionen angesprochen werden. Eine Vielzahl von IP-Cores von Xilinx sind in VHDL verfasst. Da in der SpartanMC Entwicklungsumgebung IP-Cores zum Großteil in Verilog verfasst wurden, ist es gegebenfalls notwendig, Anpassungen an der Toolchain vorzunehmen, um VHDL Designs zu unterstützen.\\
Optional wäre es wünschenswert, die erfolgreiche Paralellisierung eines Microblaze Programms durch \textmu\/Streams zu zeigen. Hierzu wäre es notwendig, 
\textmu\/Streams soweit anzupassen, dass eine entsprechende Hardwarekonfiguration mit den neu integrierten Komponenten erstellt werden kann und
die Aufrufe der Core-Konnektoren durch Aufrufe der FSL-Blöcke ersetzt werden.
\section{Integration in JConfig}
\subsection{Hardware}
Bevor über die Integration nachgedacht werden kann, muss zunächst festgelegt werden, welche Hardware verwendet wird. Über die Jahre hat Xilinx diverse Versionen des Microblaze veröffentlicht. Die Implementationen des Prozessors sind allesamt verschlüsselt und können nur mit entsprechendem Key entschlüsselt werden. Mit der Einführung von Vivado und der Einstellung der Entwicklung von ISE, änderte sich auch die Art der Verschlüsselung, sodass neuere Versionen des Microblaze nicht mehr von der alten Toolchain entschlüsselt werden können. Da die JConfig Toolchain allerdings Gebrauch von der ISE Toolchain macht, kommen für diese Arbeit nur der Microblaze v8.50c oder ältere Versionen in Frage.\\
Ausgehend vom ISE Installationsverzeichnis, ist das Verzeichnis für die IP-Cores an folgender Stelle zu finden: \textit{"14.7/ISE\_DS/EDK/hw/XilinxProcessorIPLib/pcores/"}. Dort sind alle Versionen des Microblaze, sowie alle Versionen der restlichen Hardware abgelegt. Für sämtliche verwendete IP-Cores werden die aktuellsten Versionen verwendet.
\subsection{Hardwarebeschreibung}
\subsubsection{Erzeugung der Hardwarebeschreibung mit XPS}
Um den Microblaze in JConfig integrieren zu können, ist es zunächst notwendig, eine Hardwarebeschreibung zu erstellen, die den Microblaze instanziiert und parametrisiert.
Diese kann entweder manuell erstellt werden oder mit XPS. In XPS kann dazu mit dem Base-System-Builder ein einfaches System erzeugt werden. Ein Beispiel für ein System ohne Peripherie ist in Abbildung \ref{fig:XPS_EXAMPLE} zu sehen.
\begin{figure}[ht!]
\centering
\includegraphics[width=1\linewidth]{./bilder/XPS_EXAMPLE}
\caption{Einfaches mit XPS generiertes System ohne Peripherie}
\label{fig:XPS_EXAMPLE}
\end{figure}
Das System besteht aus einem Micorblaze, zwei LMB, zwei Speichercontrollern (je einen für Daten und einen für Instruktionen), einem Block RAM als Programm- und Datenspeicher, einem AXI4-Light-Bus zur Anbindung von Peripherie, einem Reset-Core, einem Taktgenerator und einem Debug Modul. Um die Komplexität des Systems und somit zusätzliche Fehlerquellen zu reduzieren, werden der AXI4-Light-Bus, der Taktgenerator und das Debug Modul zunächst entfernt. Der AXI4-Bus wird nicht benötigt, da es zunächst in der SpartanMC Entwicklungsumgebung mit dem UART IP-Core nur eine Peripherie geben wird und diese direkt mit dem Prozessor verbunden werden kann. Der Taktgenerator wird nicht benötigt, da JConfig bereits über eigene Taktgeneratoren verfügt und das Debug Modul wird nicht benötigt, um die grundsätzliche Lauffähigkeit eines Microblaze Systems zu gewährleisten, sondern nur um Softwareanwendungen zu debuggen.\\
Nun kann in XPS eine Netzliste erzeugt werden. Als Nebenprodukt werden Wrapperdateien erstellt, die die einzelnen Komponenten mit den, im XPS angegebenen Einstellungen instanziieren. Die erzeugten Dateien werden nach Möglichkeit in der präferierten Sprache erzeugt, allerdings funktioniert dies bei manchen Wrapper Dateien nicht und sie werden in VHDL generiert. Die Top-Level-Beschreibung, welche sämtliche Komponenten instanziiert und miteinander verbindet, kann allerdings auch in Verilog erzeugt werden. Desweiteren besitzt die Top-Level-Beschreibung lediglich die, im XPS als extern makierten Signale als Eingänge und Ausgänge. Die Wrapper Datei für das Block RAM stellt eine Besonderheit dar. Für das Block RAM wird nämlich, entsprechend der angegebenen Größe des Speichers, ein elaboriertes Modell erzeugt. Dieses Modell funktioniert nur für die angegebene Speichergröße und lässt sich ohne großen Aufwand auch nicht ändern. Als Lösung des Problems wird ein generisches Speichermodul in Verilog geschrieben, welches das Verhalten der elaborierten Blöcke nachahmt (siehe \ref{subsubsec:genMem}).\\
Um nun eine konfigurierbare Hardwarebeschreibung zu erhalten, ist es notwendig, den Wrapperdateien und der Top-Level-Beschreibung Parameter hinzuzufügen. So können bei der Instanziierung des Top-Level-Moduls Parameter übergeben werden, die dann wiederum bei der Instanziierung der einzelnen Komponenten des Systems an diese weitergegeben werden können. Ebenso ist es erforderlich, dass Signale die verwendet werden sollen, als Eingänge bzw. Ausgänge der Top-Level-Beschreibung hinzugefügt und über Signale mit der entsprechenden Komponente verbunden werden. Für den Microblaze sind dies zu Beginn, neben Takt- und Reset-Signalen, Signale für den Programm- und Datenspeicher, AXI4-Signale und Signale, die dem FSL-Interface zuzuschreiben sind.\\
Unter Berücksichtigung der XML-Modulbeschreibung, ist es notwendig, den Speicher als externes Modul zu handhaben, da die XML-Syntax für Prozessoren keine integrierten Speicher unterstützt. Hierzu werden die Speichercontroller und das Block RAM in einem weiteren Modul zusammengefasst (siehe \ref{subsubsec:genMem}), sodass das Modul für den Microblaze noch aus den Instanzen für den Microblaze selbst, den beiden LMB und dem Reset-Core besteht. Die LMB-Cores erlauben es mehr als einen Speicher an den Microblaze anzuschließen.\\
Welche Parameter für die Konfiguration des Microblaze wichtig sind, wird im nächsten Abschnitt behandelt. 
\subsubsection{Handhabung der Parameter}
In diesem Kapitel wird behandelt, welche Parameter für die Konfiguration des Microblaze-Systems wichtig sind.
Für die Parameter des Microblaze wird die Tabelle aus dem Abschnitt \textit{MicroBlaze Core Configurability} in Kapitel 3 des Microblaze Datenblattes \cite{MBREF} herangezogen. Parameter, die in der Tabelle nur einen Wert in der Spalte \textit{Allowable Values} haben, werden fest auf diesen Wert gesetzt und können nicht vom Anwender verändert werden. In der Tabelle \ref{tab:MicParam} werden lediglich die Parameter behandelt, welche mehr als einen erlaubten Wert haben, allerdings trotzdem auf einen festen Wert gesetzt werden. Alle anderen Parameter, die nicht genannt werden, stehen dem Nutzer zur freien Konfiguration zur Verfügung. Erklärungen zu den Bedeutungen der einzelnen Parametern werden in Kurzform in der XML-Modulbeschreibung hinterlegt, sodass der Nutzer während der Erstellung eines Systems weiß, welche Funktion die einzelnen Parameter haben. Für detailliert Beschreibungen sei noch einmal auf das Datenblatt des Microblaze verwiesen \cite{MBREF}.
\begin{table}[ht!]
	\begin{tabular}{|l|c|p{10cm}|}
		\hline \textbf{Parameter} & \textbf{Wert} & \textbf{Beschreibung} \\ 
		\hline C\_LOCKSTEP\_SLAVE & 0 & Im Lockstep Modus können zwei oder mehrere \newline Microblaze das selbe Programm ausführen und die Ergebnisse miteinander vergleichen, um Hardwarefehler zu erkennen. Wird nicht benötigt.\\ 
		\hline C\_ENDIANNESS & 1 & Endianness wird auf Little Endian festgelegt. In XPS lässt sich dieser Wert nicht manuell verändern, dementsprechend wird auch unter JConfig darauf verzichtet.\\ 
		\hline C\_D\_AXI & 1 & Das AXI-Daten-Interface kann verwendet werden. \\ 
		\hline C\_I\_AXI & 1 & Das AXI-Instruktionen-Interface kann verwendet werden. (Nützlich bei externen Speicher) \\ 
		\hline C\_D\_PLB & 0 & Da bereits AXI verwendet wird, wird der PLB Bus nicht gebraucht und deshalb deaktiviert. \\ 
		\hline C\_I\_PLB & 0 & Siehe vorheriger Eintrag. \\ 
		\hline C\_D\_LMB & 1 & Das LMB-Daten-Interface kann verwendet werden. \\ 
		\hline C\_I\_LMB & 1 & Das LMB-Instruktionen-Interface kann verwendet werden. \\ 
		\hline C\_IPLB\_BUS\_EXCEPTION & 0 & Da der PLB nicht verwendet wird, werden die Exceptions auch nicht benötigt. \\ 
		\hline C\_DPLB\_BUS\_EXCEPTION & 0 & Siehe vorheriger Eintrag. \\ 
		\hline C\_ADDR\_TAG\_BITS & 17 & Das Datenblatt gibt kaum nützliche Informationen zu diesem Parameter, daher wird er auf den Default-Wert festgesetzt. \\ 
		\hline C\_DCACHE\_ADDR\_TAG & 17 & Siehe vorheriger Eintrag. \\ 
		\hline C\_USE\_EXT\_BRK & 0 & Siehe vorheriger Eintrag. \\ 
		\hline C\_USE\_EXT\_NM\_BRK & 0 & Siehe vorheriger Eintrag. \\ 
		\hline 
	\end{tabular}
	\centering
	\caption{Parameter des Microblaze die mehr als einen erlaubten Wert haben, aber dennoch auf einen Wert festegelegt werden.}
	\label{tab:MicParam}
\end{table}
Neben dem Microblaze werden mit einem generischen Speichermodul, einem UART-Modul und einem FSL-Modul noch drei weitere Hardwarekomponenten für JConfig erstellt, die parametrisiert werden können. Welche Parameter zur Verfügung stehen und was diese bewirken, ist in den Tabellen \ref{tab:MemParam}, \ref{tab:UARTParam} und \ref{tab:FSLParam} aufgeführt.

\begin{table}[ht!]
	\begin{tabular}{|l|p{10cm}|}
		\hline \textbf{Parameter} & \textbf{Beschreibung} \\ 
		\hline C\_FAMILY & Dieser Parameter wird von Xilinx IP-Cores verwendet, um, je nach verwendeter Hardware, vorhandene Primitives zu instanziieren. \\ 
		\hline RAMBLOCKS & Ähnlich wie für den Speicher des SpartanMC, kann angegeben werden, wie viel RAM instanziiert werden soll. Ein Block ist jeweils 2KB groß. Es können nur Größen gewählt werden, die einer Potenz von Zwei entsprechen, da mit jeder Vergrößerung des Speichers sich die Anzahl der Bitlanes verdoppelt.\\ 
		\hline C\_BASEADDRESS & Gibt die Startaddresse des Speichers an.\\ 
		\hline 
	\end{tabular}
	\centering
	\caption{Parameter des Speichermoduls für den Microblaze.}
	\label{tab:MemParam}
\end{table}

\begin{table}[ht!]
	\begin{tabular}{|l|p{10cm}|}
		\hline \textbf{Parameter} & \textbf{Beschreibung} \\ 
		\hline C\_S\_AXI\_ACLK\_FREQ\_HZ & Frequenz des Taktsignals, welches automatisch aus dem Takt abgeleitet wird.\\ 
		\hline C\_FAMILY & Dieser Parameter wird von Xilinx IP-Cores verwendet, um, je nach verwendeter Hardware, vorhandene Primitives zu instanziieren.\\ 
		\hline C\_BAUDRATE & Für die Übertragung verwendete Baudrate. Eine Auswahl von Werten zwischen 110 und 921600 steht zur Verfügung. \\ 
		\hline C\_DATABITS & Anzahl der zu übertragenden Bits pro Frame. Werte von 5 bis 8 sind möglich. \\ 
		\hline C\_USE\_PARITY & Gibt an, ob ein Paritätsbit mit übertragen wird. \\ 
		\hline C\_ODD\_PARITY & Gibt an, ob die Parität gerade oder ungerade ist. \\ 
		\hline 
	\end{tabular}
	\centering
	\caption{Parameter des UART-Cores.}
	\label{tab:UARTParam}
\end{table}

\begin{table}[ht!]
	\begin{tabular}{|l|p{10cm}|}
		\hline \textbf{Parameter} & \textbf{Beschreibung} \\ 
		\hline C\_EXT\_RESET\_HIGH & Gibt an, ob das externe Reset-Signal high-aktiv oder low-aktiv ist.\\ 
		\hline C\_ASYNC\_CLKS & Gibt an, ob das FSL-Interface synchron oder asyncron betrieben werden soll.\\ 
		\hline C\_IMPL\_STYLE & Wenn dieser Parameter auf 0 gesetzt ist, wird der FIFO-Speicher mit LUT RAMs implementiert, andernfalls mit Block RAMs.   \\ 
		\hline C\_USE\_CONTROLL & Bestimmt, ob das Steuerbit mit übertragen wird oder nicht.\\ 
		\hline C\_FSL\_DWIDTH & Ist auf 32 Bit festgelegt, da andere Datenbreiten im Zusammenhang mit dem Microblaze wenig sinnvoll sind. \\ 
		\hline C\_FSL\_DEPTH & Beschreibt die Größe des FIFO-Speichers. Diese ist von den Parametern C\_ASYNC\_CLKS und C\_IMPL\_STYLE abhängig. C\_ASYNC\_CLKS=0 => 1-8192. C\_ASNYNC\_CLKS=1 und C\_IMPL\_STYLE=0 => 16-128. C\_ASNYNC\_CLKS=1 und C\_IMPL\_STYLE=1 => 512-8192. Ist C\_ASNYNC\_CLKS=1 muss die Größe außerdem einer Potenz von Zwei entsprechen.\\ 
		\hline C\_READ\_CLOCK\_PERIOD & Wird automatisch aus dem Taktsignal berechnet. Wird genutzt um Timing Constraints für den asynchronen Pfad durch LUT RAMs zu erzuegen \\
		\hline 
	\end{tabular}
	\centering
	\caption{Parameter des FSL-Cores.}
	\label{tab:FSLParam}
\end{table}

\subsubsection{Implementation eines generischen Speichermoduls} \label{subsubsec:genMem}
Bevor über eine Implementation eines generischen Speichermoduls nachgedacht werden kann, sollte zunächst die Struktur, des von Xilinx verwendeten Speichers erläutert werden. Hierzu wird Abbildung \ref{fig:XilinxMEM} herangezogen.
\begin{figure}[ht!]
\centering
\includegraphics[width=0.7\linewidth]{./bilder/XilinxMEM}
\caption{Speicherstruktur der von Xilinx generierten Speichermodule}
\label{fig:XilinxMEM}
\end{figure}
In der Abbildung sind einfache Blockschaltbilder für einen 4KByte und einen 8KByte großen Speicher dargestellt. Angefangen bei einer Mindestgröße von 2KByte verdoppelt sich die Anzahl der Block RAMs bei jeder Vergrößerung. Dabei halbiert sich jeweils die Größe der Bitlanes bis hin zu einem Minimum von eins bei einer maximalen Speichergröße von 64KByte. Die vom XPS generierten Modelle instanziieren immer eine fixe Anzahl an Block RAMs, sodass es nicht möglich ist, mit der selben Methode wie beim Microblaze eine generische Parametriserbarkeit herzustellen. Daher muss, auf Grundlage der elaborierten Modelle, ein generisches Speichermodul geschrieben werden, dessen Größe über einen Parameter festgelegt werden kann.\\
Um dies zu erreichen, wird ein Verilog-Modul geschrieben, welches die Parameter \textit{RAMBLOCKS} und \textit{C\_FAMILY} hat. Über eine for-Schleife in einer generate-Umgebung werden entsprechend des Parameters \textit{RAMBLOCKS} dual-ported Block RAMS instanziiert. Einen Port handhabt Daten und der andere Instruktionen. Eine Funktion berechnet dabei aus dem Parameter die Breite der einzelenen Bitlanes. Um die Signalverbidungen zu realisieren, werden wires und regs deklariert, deren Größe mit \textit{RAMBLOCKS} skaliert. Über Indexing wird innerhalb der for-Schleife die Realisierung der einzelnen Bitlanes vorgenommen. Für das 4-Bit breite Write-Enable-Signal wird zusätzliche Kombinatorik hinzugefügt, da die Handhabung nicht durch einfaches Indexing umgesetzt werden kann.\\
Zuletzt werden der Speicher und zwei Speichercontroller in einer Top-Level-Beschreibung instanziiert und miteinander verbunden.

\subsection{Modulbeschreibungen} \label{subsec:Moduldesc}
Für den Microblaze, den Speicher, den UART-Core und den FSL-Core müssen nun XML-Modulbeschreibungen erstellt werden. Nachfolgend werden alle hierzu notwendigen Schritte besprochen:
\begin{itemize}
\item \textbf{Microblaze}: Der Microblaze wird in der Beschreibung als Prozessor kategorisiert. Neben den Wrapperdateien für den Microblaze, den LMBs und dem Reset-Modul müssen noch Pfadangaben für die HDL-Dateien der Implementationen dieser Cores angegeben werden. Um dies umzusetzen, wird die XML-Syntax mit \textit{externalRoot} um ein neues Konstrukt erweitert. Dieses Konstrukt erlaubt es absolute Pfade außerhalb des Default-Verzeichnisses anzugeben und wurde nicht im Rahmen dieser Arbeit ergänzt, sondern von einem Kommilitonen, der zum Zeitpunkt dieser Arbeit für die Wartung von JConfig verantwortlich war. Alle Dateien, die sich im Scope eines \textit{externalRoot}-Konstruktes befinden, verwenden den angegeben Pfad als Ausgangsverzeichnis. Im Rahmen dieser Arbeit, wurde in der Klasse \textit{HDLDescription} im Package \textit{de.tu\_darmstadt.rs.spartanmc.devxml.descriptions.generic} eine Anpassung vorgenommen, die Umgebungsvariablen in Pfadangaben auflöst. So wird über die Umgebungsvariable \textit{XILINX\_ROOT} plattformunabhängig der Pfad zu den HDL-Dateien angegeben. Desweiteren ist es möglich, einen Namen für eine VHDL-Library anzugeben, unter dem die entsprechenden Dateien zusammengefasst werden. Dies ist für die Xilinx Cores notwendig, da innerhalb der Implementationen einige Libraries referenziert werden. Außerdem verkürzen Libraries die Kompilierzeit während der Simulation.\\
Es folgen daraufhin die Angaben der Parameter. Hierbei wird darauf geachtet, dass die Gruppierung der Parameter in ähnlicher Weise erfolgt, wie es im Wizard des XPS der Fall ist. Die Parameter werden desweiteren mit einem Beschreibungstext versehen, sodass sich die Funktion für den Anwender bei der Parametrisierung erschließt. Außerdem haben die Konfigurationsmöglichkeiten einiger Parameter eine relativ geringe Aussagekraft. So lässt sich beispielsweise der Modus der MMU über einen Parameter steuern, der die Werte null bis drei annehmen kann. Um mehr Informationsgehalt zu vermitteln, wurde ein String-Parameter eingeführt, der die Werte \textit{``None''}, \textit{``Usermode''}, \textit{``Protection''} und \textit{``Virtual''} annehmen kann. Innerhalb der Top-Level-Verilog-Beschreibung wird dann über eine Aliasing-Funktion der String zu einem Integer umgewandelt. Dies wird für weitere Parameter durchgeführt, bei denen die Aussagekraft der Auswahlmöglichkeiten zu wenig Information vermittelt.\\
Desweiteren schließen sich einige Konfigurationsmöglichkeiten gegenseitig aus. So ist es beispielsweise nicht möglich, die MMU zu verwenden, wenn der Parameter \textit{C\_AREA\_OPTIMIZED} gesetzt ist. Um eine Auswahl in einem solchen Fall zu verhindern, wird das Konstrukt \textit{relevantIf} in Zusammenhang mit einer invertierten Version von \textit{C\_AREA\_OPTIMIZED} verwendet.\\
Darauf folgen Bus-Deklarationen für Speicherbusse und FSL-Busse (siehe \ref{subsec:BusDesc}), wie auch Signal-Deklarationen für Takt, Reset und dem AXI-Bus.\\
Abschließend wird das \textit{addressLayout} für den Daten- und Instruktionsspeicher definiert. Für Peripherie wurde noch kein Adressraum festgelegt, da es bis lang nur den UART-Core als Peripherie gibt, und dieser direkt mit dem Microblaze verbunden wird. Sobald ein AXI-Bus in JConfig verfügbar ist, muss auch ein Adressraum für die Peripherie hinzugefügt werden.
\item \textbf{Speicher}: Die Erstellung der Modulbeschreibung für den Speicher erfolgt ähnlich wie für den Microblaze. Die verfügbaren Parameter sind in Tabelle \ref{tab:MemParam} aufgelistet. Der Speicher wird allerdings als Memory kategorisiert und hat mit dem \textit{memory}-Konstrukt eine Besonderheit. Mit \textit{memory} wird die Größe des Speichers, das Namensschema der Speicherinstanzen und die Reset-Werte beschrieben. Außerdem kann angegeben werden, ob der Speicher eine Aufteilung zwischen Daten und Instruktionen vorsieht. Diese Informationen werden später bei der Erzeugung der Memory-Map benötigt.
\item \textbf{UART}: Auch hier wird die Modulbeschreibung nach ähnlichem Muster erstellt. Der UART-Core ist allerings als Peripherie kategorisiert. Über das \textit{registers}-Konstrukt können Angaben zu Registern innerhalb des Cores gemacht werden, die später während der Erstellung der Software verwendet werden können. Im Rahmen dieser Arbeit konnte dieses Konstrukt für die UART allerdings nicht mehr erstellt werden. Die zur Verfügung stehenden Parameter sind in Tabelle \ref{tab:UARTParam} zu finden.
\item \textbf{FSL}: Der FSL-Core wird als Core Interconnect kategorisiert und hat keine weiteren Besonderheiten. Die Parameter sind in Tabelle \ref{tab:FSLParam} aufgelistet.
\end{itemize}

\subsection{Busbeschreibungen}\label{subsec:BusDesc}
Um das Verbinden einzelner Komponenten zu vereinfachen, werden zusammengehörige Signale zu einem Bus zusammengefasst. Im Folgenden wird auf die, im Rahmen dieser Arbeit erstellten Busbeschreibungen eingegangen:
\begin{itemize}
\item \textbf{FSL-Busse}: Für den FSL-Bus werden zwei neue Busbeschreibungen in Form von XML-Dateien erstellt. Ein Bus beschreibt die Verbindungen für das Master-Interface und der andere die Verbindungen für das Slave-Interface (siehe Abbildung \ref{fig:FSLBSB}). Die Busbeschreibungen enthalten Defintionen für \textit{masterPorts} und \textit{slavePorts}, sowie alle dazugehörigen Signale, deren Richtung und Datenbreite.
Bei der Deklaration innerhalb der Modulbeschreibung für den Microblaze wird darauf geachtet, das die einzelnen Signale in den Busbeschreibungen mit der Option \textit{combineUsing="parallel"} versehen sind. Dies bewirkt, dass die einzelnen Signale, der bis zu 16 FSL-Busse konkateniert und so an den Microblaze weitergeleitet werden.
\item \textbf{Speicherbusse}: Sowohl alle für den Datenport des Speichers relevanten Signale, sowie alle für den Instruktionsport relevanten Signale, werden zu je einem Bus zusammengefasst. Die Beschreibung erfolgt nach ähnlichem Schema wie bei den FSL-Bussen. Die \textit{masterPorts} fassen alle Signale auf Seiten des Microblaze zusammen, während \textit{slavePorts} alle Signale des Speichers umfasst.
\end{itemize}
Für die AXI-Signale wurden im Rahmen dieser Arbeit noch keine Busse erstellt, dies sollte allerdings mit der Integration eines AXI-Bus-IP-Cores nachgeholt werden.

\section{Integration in die SpartanMC Toolchain}
\subsection{Ausgangslage}
Der Microblaze und alle notwendigen Peripherien sind in JConfig eingebunden. Mit den von JConfig generierten Dateien, kann die Toolchain allerdings noch kein funktionsfähiges Bitfile erzeugen, da die Speicherarchitekturen des SpartanMC und des Microblaze verschieden sind. Der Speicher des SpartanMC ist so aufgebaut, dass dem Parameter \textit{RAMBLOCKS} entsprechend Block RAMs instanziiert werden, dessen Bitlanes immer 18-Bit breit sind. So wird bei jedem Speicherzugriff auf genau ein Block RAM zugegriffen und nicht auf alle, wie es beim Microblaze der Fall ist.\\
Das Tool Initramj der SpartanMC Entwicklungsumgebung wird zur Speicherintialisierung für den SpartanMC verwendet und ist auf die vorher beschriebene Speicherstruktur angepasst. Das Tool ist unter anderem in der Lage, Speicherinitialisierungen in Form von UCF-Dateien, Verilog-Dateien oder Bitfile-Updates durchzuführen. Hierzu benötigt es eine ELF-Datei und eine Memory-Map in Form einer Datei namens \textit{memory.xml}. Es wäre also denkbar, Initramj dahingehend anzupassen, dass es auch die Speicherinitilaisierung für die 32-Bit-Speicherarchitektur des Microblaze durchführen kann. Dies hätte den Vorteil, dass die Speicherininitialisierung gekapselt von einem einzigen Tool gehandhabt würde und somit kaum Änderungen an den Makefiles notwendig wären.\\
Eine Alternative zu Initramj stellt die Speicherinitialisierung mit dem Xilinx-Tool data2mem dar. Hierfür müsste JConfig allerdings eine weitere Memory-Map in Form einer BMM-Datei erzeugen. Desweiteren müssten die Makefiles angepasst werden, sodass sie data2mem aufrufen. Im Rahmen dieser Arbeit wurde die Variante mit data2mem umgesetzt, da der Aufwand zur Anpassung von Initramj als zu hoch eingeschätzt wurde.\\
Desweiteren ist es notwendig, den Microblaze GCC in die Toolchain zu integrieren, da der SpartanMC Compiler keinen, vom Microblaze ausführbaren Code generieren kann. Um dies zu erreichen, müssen weitere Änderungen an den Makefiles vorgenommen werden.
\subsection{Erzeugung der BMM-Datei}
Damit JConfig in der Lage ist BMM-Dateien zu erzeugen, muss die Software erweitert werden. Das Package \textit{de.tu\_darmstadt.rs.spartanmc.jconfig.generation.outputProvider} des Tools libjconfig enthält diverse Klassen, welche auf Basis der, über das UI eingegebenen Informationen, die von der Toolchain benötigten Dateien generieren. An dieser Stelle wird eine neue Klasse namens \textit{BMMBuilderProvider} hinzugefügt. Diese Klasse generiert für jeden Speicher, der an einem Microblaze angeschlossen ist, einen Eintrag in der BMM-Datei, gemäß der Syntax, die in Abschnitt \ref{subsec:MemInitBMM} beschrieben ist. Das Namensschema für die Speicherinstanzen wird aus der XML-Modulbeschreibung des Speichers gewonnen. Um festzustellen, ob ein Speicher an einem Microblaze oder an einem SpartanMC angeschlossen ist, werden den XML-Modulbeschreibungen zwei virtuelle Parameter hinzugefügt (\textit{MB\_FLAG} für den Microblaze und \textit{SPMC\_FLAG} für den SpartanMC). Der \textit{BMMBuilderProvider} fügt einen Speicher nur dann zur Memory-Map hinzu, wenn der Parameter \textit{MB\_FLAG} vorhanden ist. Auf der anderen Seite muss aber auch verhindert werden, dass der Microblaze-Speicher in die \textit{memory.xml} aufgenommen wird, da Initramj ansonsten versuchen würden, eine Speicherinitialisierung für diese Module ebenfalls durchzuführen. Daher wurde die Klasse \textit{MemoryMapBuilderProvider} dahingehend angepasst, dass nur Speicher mit dem Parameter \textit{SPMC\_FLAG} hinzugefügt werden.\\
Damit JConfig die BMM-Datei auch erstellt, muss eine Instanz des \textit{BMMBuilderProvider} im Konstruktor der Klasse \textit{XilinxIseDocumentBuilder} hinzugefügt werden. Des weiteren werden die Dateien \textit{memory.xml} bzw. \textit{memory.bmm} nur dann erstellt, wenn auch ein SpartanMC bzw. ein Microblaze im System vorhanden ist.

\subsection{Anpassungen in der Toolchain zur Speicherinitialisierung}
Mit der BMM-Datei kann data2mem nun verwendet werden (auch wenn die ELF-Dateien mit dem Microblaze noch nicht konform sind, kann data2mem diese verwenden). Die einzelnen Anpassungen an den Makefiles werden im nachfolgenden beschrieben.
\subsubsection{Speicherinitialisierung mit UCF-Datei}
Die Speicherinitialisierung mit einer UCF-Datei findet statt, wenn ein Bitfile über die Regel \textit{"all"} erzeugt werden soll. Dabei wird eine weitere Regel namens \textit{\$(UCF)} aufgerufen, welche dafür sorgt, dass alle relevanten UCF-Dateien aktuell sind und diese dann über das Shell-Skript \textit{mkucf} zu einer Datei zusammengefasst werden. Um UCF-Dateien für Microblaze-Prozessoren einzubinden, wird eine neues Makefile mit dem Namen \textit{firmware\_mb.mk} erstellt, welches die Erzeugung der UCF-Dateien handhabt. Da data2mem pro Aufruf allerdings nur eine einzige Firmware behandeln kann, muss das Tool mehrfach aufgerufen werden. Hierzu wird eine \textit{foreach}-Schleife über alle Firmwares des Microblaze durchlaufen. Diese erzeugt für jede Firmware mit data2mem ein UCF-File. Um zu unterscheiden, welche Firmware zu welcher Prozessorart gehört, wird die Konstante \textit{JCONFIG\_FIRMWARE\_IDS} aus dem ,von JConfig generierten Makefile \textit{firmwares.mk} aufgeteilt in \textit{JCONFIG\_SPARTANMC\_FIRMWARE\_IDS} und \textit{JCONFIG\_MICROBLAZE\_FIRMWARE\_IDS}. Die hierzu notwendigen Änderungen an der Software werden in der Klasse \textit{LegacyMkBuilderProvider} vorgenommen.\\
Die so erstellten UCF-Dateien werden an das Shell-Skript \textbf{mkucf} übergeben, welches daraus eine einzelne Datei macht. Die Syntax für den Aufruf von data2mem ist wie folgt:\\\\
\indent
\textit{data2mem -bm <BMM-Datei> -bd <ELF-Datei> tag <Prozessorname> -o u <UCF-Datei>}\\\\
Die Option \textit{-bm} spezifiziert die BMM-Datei und \textit{-bd} die ELF-Datei. Der Zusatz \textit{tag} ist notwendig, um die zum Prozessor passende \textit{ADDRESS\_MAP} in der BMM-Datei zu spezifizieren und \textit{-o u} gibt an, dass das Ausgabeformat eine UCF-Datei sein soll.
\subsubsection{Aktualisierung des Bitfiles}
Die Speicherinhalte eines bestehenden Bitfiles werden beim Aufruf der Regel \textit{bitgen} erzeugt. Um die Updates für die Speicher des Microblaze durchzuführen, wird in die Liste der Prerequisites eine neue Regel namens \textit{bit\_mb\_update} hinzugefügt. Diese Regel ruft für jede Firmware eines Microblaze data2mem auf und aktualisiert das Bitfile. Für die Aktualisierung des Bitfiles benötigt data2mem allerdings eine BMM-Datei, die Placement-Informationen der Speicherinstanzen beinhaltet. Diese Datei kann allerdings von der Xilinx Toolchain erzeugt werden, wenn dem Translate-Tool NGDBuild beim Aufruf die originale BMM-Datei mit der Option \textit{-bm} übergeben wird. Die Syntax für den Aufruf von data2mem ist wie folgt:\\\\
\indent
\textit{data2mem -bm <BMM-Datei> -bd <ELF-Datei> tag <Prozessorname>\\
\indent \indent \indent -bt <zu aktualisierendes Bitfile> -o b <neues Bitfile>}
\subsubsection{Speicherintialisierung für die Simulation}
Die Speicherinitialisierung für die Simulation wird mittels Verilog-Dateien durchgeführt. Durch ausführen der Regel \textit{sim} wird eine entsprechende Verilog-Datei von Initramj erzeugt. Durch hinzufügen einer Regel in dem Makefile \textit{firmware\_mb.mk} wird durch Aufruf von data2mem für jede Firmware eine Verilog-Datei erzeugt, die die Speicherinitialisierung enthält. Der Aufruf hierfür ist ähnlich, wie bei der Erzeugung der UCF-Dateien:\\\\
\indent
\textit{data2mem -bm <BMM-Datei> -bd <ELF-Datei> tag <Prozessorname> -o v <Verilog-Datei>}\\\\
Die so erzeugten Verilog-Dateien müssen nun noch zusammengefasst werden, da innerhalb der automatisch erzeugten Testbench nur ein einziges Initialisierungsmodul instanziiert wird. Es könnte auch eine Anpassung der Software vorgenommen werden, die für jede erzeugte Verilog-Datei eine Instanz in der Testbench erzeugt. Das Zusammenfassen der einzelnen Dateien stellt allerdings die elegantere Lösung dar und wird deshalb umgesetzt. Hierzu wurde ein Shell-Skript namens \textit{mksim} auf Basis des Skripts \textit{mkucf} erstellt. Es erhält als Eingabe eine unbestimmte Anzahl an Verilog-Dateien (sowohl für SpartanMC als auch Microblaze) und fügt diese zu einer einzigen Datei zusammen. Dabei werden noch zwei Zeilen eingefügt, die die Speicherinitialisierung als Verilog-Modul definieren (die entsprechenden Zeilen werden bei der Datei für den SpartanMC vorher entfernt). Außerdem wird der hierarchische Name der Speicherinstanzen, als Konsequenz aus der Initialisierung innerhalb der Testbench, um das Präfix \textit{``UUT.''} erweitert.\\
Der Aufruf des Skripts findet innerhalb der Regel \textit{sim} statt.

\subsection{Unterstützung von Libraries in der Simulation}
In Abschnitt \ref{subsec:Moduldesc} wurde bereits darauf eingegangen, dass Xilinx IP-Cores VHDL-Libraries referenzieren. Einer der Vorteile von Libraries ist, dass alle Dateien innerhalb einer vorkompilierten Library bei erneutem Kompilieren nicht berücksichtigt werden und daher Zeit gespart wird. Xilinx bietet mit dem Tool compxlib eine Möglichkeit Simulationslibraries für sämtliche IP-Cores des EDK zu erstellen. Die Auswahlmöglichkeiten bei der Erzeugung sind allerdings eingeschränkt. So ist es nicht möglich einzelne IP-Cores zu kompilieren, sondern es werden Simulationsmodelle für alle Cores des EDK erstellt. Der hierfür verwendete Speicherplatz ist mit mehr als 2GByte verhältnismäßig groß. Daher werden nur die Simulationslibraries der Hardware behalten, welche in JConfig eingebunden wurde. Die Auswahl wird mit einer Größe von ca. 70MByte dem SpartanMC Git-Repository hinzugefügt, um zu gewährleisten, dass Nutzer die Libraries nicht selbst kompilieren müssen.\\
Um die neuen Libraries in der Simulation einzubinden, müssen Anpassungen an dem Skript \textit{configure.tcl} vorgenommen werden. Hierzu benötigt das Skript Informationen darüber, welche Libraries eingebunden werden müssen und wo diese zu finden sind. In der Klasse \textit{XilinxIseProjectBuilderProvider} wird dazu ein neues Makefile namens \textit{sim\_lib.mk} erzeugt, welche eine Liste aller benötigten Libraries definiert. Desweiteren wird die Erstellung des Do-Skripts \textit{spartanmc\_worklib.fdo} angepasst, welche Befehle zur Kompilierung aller notwendigen Dateien enthält, sodass Dateien innerhalb einer Library nicht hinzugefügt werden, da diese ja vorkompiliert sind. Um die Pfadangabe zu den Simulationslibraries zu realisieren, wird in JConfig der Konfiguration unter dem Reiter Target das Feld \textit{Simulation Library Path} hinzugefügt. Diese Pfadangabe wird im Makefile \textit{project.mk} unter der Konstanten \textit{JCONFIG\_SIMULATION\_LIBRARY\_PATH} abgelegt. Mit diesen Informationen kann das Skript \textit{configure.tcl} die entsprechenden Libraries hinzufügen. Des weiteren wurden noch Mappings für die Libraries \textit{ieee}, \textit{std} und \textit{synopsys} ergänzt, da diese ebenfalls benötigt werden.
\subsection{Integration des Microblaze GCC}
Die Integration des Microblaze GCC konnte im Rahmen dieser Arbeit aus zeitlichen Gründen nicht durchgeführt werden. Es wird an dieser Stelle allerdings kurz darauf eingegangen, welche Ansätze für die Integration vorstellbar wären.\\
Die für den Ablauf des Build-Prozesses verwendeten Makefiles sind, ausgehend vom SpartanMC Root-Verzeichnis, unter dem Pfad \textit{``src/scripts/make/firmware/''} zu finden. Das Makefile \textit{fwbuild-gcc.mk} ist dabei für die Erzeugung der Firmware verantwortlich. Hierzu erstellt es auf Basis des Templates \textit{gcc-build.mk} (zu finden im Verzeichnis \textit{``templates''}) je ein Makefile für jede Firmware. Diese erzeugten Makefiles beinhalten alle notwendigen Regeln, um eine ELF-Datei zu erzeugen. In \textit{fwbuild-gcc.mk} werden desweiteren alle verwendeten Tools, Source-Dateien und Flags für Compiler, Assembler und Linker festgelegt. Um den Microblaze GCC zu integrieren sollten in dieser Datei Änderungen vorgenommen werden.\\
Der Microblaze GCC und die Binutils sind im Xilinx ISE Installationsverzeichnis unter folgendem Pfad zu finden: \textit{``14.7/ISE\_DS/EDK/gnu/microblaze/lin/bin/''}. Desweiteren müssen basierend auf der Konfiguration des Microblaze bestimmte Compiler-Flags gesetzt werden, um Hardwarebeschleuniger zu integrieren. Hierzu ist es notwendig, dass die Software erkennt, wenn ein entsprechender Parameter gesetzt ist und das dazugehörige Flag in dem Makefile \textit{firmwares.mk} hinzufügt. Relevante Informationen zum GCC sind im \textit{Embedded System Tools Reference Manual} Kapitel 9 zu finden \cite{MGNU}.\\
Außerdem müssen entsprechend der verwendeten Hardware passende Treiber inkludiert werden. Die Software zur Verwendung von Xilinx IP-Cores ist unter folgendem Pfad zu finden: \textit{``14.7/ISE\_DS/EDK/sw/XilinxProcessorIPLib/drivers/''}.
%\include{Implementierung}
\chapter{Benutzungshinweise}
Grobe Gliederung:
<JConfig: Hinweise darauf, dass bei Verwendung der FSL-Blöcke der Parameter C\_FSL\_LINKS des Microblaze manuel auf die Anzahl der verwendeten Blöcke gesetzt werden muss.>
<Speicherinitialisierung: Da der GCC noch nicht integriert ist, muss die Speicherinitialisierung manuell erfolgen. Dafür wurden im hardware ordner des Microblaze zwei elf-files, eine readme und ein Skript hinterlegt, welches mehrere Verilog-Dateien zu einer zusammenfügt ähnlich wie das skript, was der maketoolchain hinzugefügt wurde.>
\chapter{Evaluation}
\section{Vorgehensweise}
In diesem Kapitel soll der erreichte Grad der Integration des Microblaze in die SpartanMC Entwicklungsumgebung evaluiert werden. Dabei soll nicht nur gezeigt werden, dass auf Microblaze basierende Systeme erstellt und synthetisiert werden können, sodern auch, dass die Integration die ursprüngliche Funktionalität nicht negativ beeinflusst. Hierzu soll zunächst ein einfaches System mit allen hinzugefügten Komponenten mit JConfig erstellt, anschließend mit der Toolchain ein Bitfile erzeugt werden. Außerdem sollen die Aktualisierung des Bitfiles und die Speicherinitialisierung für die Simulation ausgeführt werden. Dies soll zeigen, dass die Integration des Microblaze in JConfig erfolgreich war und dass die Anpassungen an den Makefiles zur Speicherinitialisierung ebenfalls funktionieren.\\
Desweiteren soll Software mit dem SDK von Xilinx geschrieben werden, welche sowohl die FSL-Blöcke als auch den UART-Core verwendet. Mit data2mem soll aus dem entstehenden ELF-File eine Speicherinitialisierung für die Simulation in Form einer Verilog-Datei erfolgen. Das System muss anschließend simuliert werden um die Funktionalität der einzelnen Systemkomponenten zu verifizieren.\\
Um zu zeigen, dass die Änderungen keinen negativen Einfluss auf SpartanMC-Systeme haben, soll der Workflow mit zwei weiteren Systemen durchlaufen werden. Das eine System soll nur SpartanMC-Instanzen beinhalten, während das andere sowohl einen SpartanMC, als auch einen Microblaze beinhalten.
\section{Erstellen des Testsystems mit JConfig}
Grobe Gliederung:
<Bild von JConfig-Konfiguration anzeigen und kurz erläutern, was zu sehen ist. Kurz Parametrisierung erwähnen. Erwähnen, dass das Zusammenfassen von FSL- und Speicherinterface zu Bussen die Verdrahtung erleichtert und das wiring von AXI-Signalen mühsam ist. Anmerken, dass ein groovy-script die Konfiguration des Microblaze noch verschnellern könnte. Erläuterungen er Parameter neben den Eingabefeldern macht das Parametrisieren einfacher, aliasing von manchen Parametern macht das ganze verständlicher. Ansonsten fühlt es sich ähnlich wie beim erstellen von SpartanMC-Systemen an.>
\section{Verwendung der Toolchain}
Grobe Gliederung:
<An der Nutzung der Toolchain hat sich für den Endnutzer prinzipiell nichts geändert was gut ist. Kurz darauf eingehen, welche Kommandozeilenbefehle eingegeben wurden um die Speicherintialisierung durchlaufen zu lassen. Wie wird überprüft, ob die Speicherinitialisierung erfolgreich war? Für die Simulation kurz darauf hinweisen, dass der Speicherinhalt Quatsch ist, da der MB GCC noch nicht integriert ist, aber dem Simulator das egal ist wenn er die Simulation starten möchte >
\section{Erstellen der Testsoftware und Erzeugung der Speicherinitialisierung für die Simulation}
Grobe Gliederung:
<Kurz beschreiben, wie vorgegangen wurde -> mit XPS Hardware zusammenbauen, die dem Testsystem ähnlich ist (evtl kleines Bild), dann SDK aufrufen und die Code Snippets für die beiden Prozessoren zeigen, Schritte für manuelle Speicherinitialisierung erläutern (evtl auf Kapitel Benutzungshinweise bezug nehmen).>
\section{Simulationsergebnisse}
Grobe Gliederung:
<Zeigen, dass in der Simulation der Microblaze Instruktionen erfolgreich aus dem Speicher liest, die Kommunikation über FSL funktioniert und dass das Ansprechen der UART über den AXI-Bus nicht klappt, weil aus irgendeinem Grund die Steuersignale vom Microblaze nicht gesetzt werden. Als Vergleich dann noch die Simulationsergebnisse des vom XPS erzeugten System in dem es funktioniert. Kurz drauf eingehen, was getan wurde um den Fehler zu finden und dass ich am Ende immer noch keinen Plan hab, warum der Microblaze sich weigert, die Steuersignale zu setzen.>
\section{SpartanMC- und Hybrid-Systeme}
Grobe Gliederung:
<Abschließend kurz beschreiben, wie die Systeme aussehen, dass die gleichen Tests wie oben stattgefunden haben, dass die memory.bmm im Falle des SPMC-only Systems wie erwartet nicht erzeugt wird und das im Hybrid-System die Speicher in der passenden Memory-Map auftauchen. Ansonsten klappt alles.>
\chapter{Fazit und Ausblick}
\section{Fazit}
In dieser Arbeit wurde der 32-Bit Softcore Prozessor Microblaze von Xilinx in die SpartanMC Entwicklungsumgebung integriert. Anhand der Abbildung \ref{fig:Schaubild} wird beschrieben, welche Schritte dafür erforderlich waren und zu welchem Grad der Microblaze integriert werden konnte.\\
Zu Beginn musste der Prozessor zusammen mit einem Speichermodul, einem FSL-Modul und einem UART-Core in den Systembuilder JConfig integriert werden. Hierzu wurden für die neue Hardware XML-Modulbeschreibungen erstellt, welche sämtliche Parameter und Signale der Komponenten definieren. Desweiteren wurden Wrapper-Module in Verilog und VHDL erstellt und der SpartanMC Hardware-Library hinzugefügt. Diese Wrapper instanziieren die neuen Cores mit den über JConfig getroffenen Einstellungen. Außerdem wurde ein neuer, generischer Speicher für den Microblaze in Verilog geschrieben, da kein generisches Modell vorhanden war. JConfig wurde dahingehend angepasst, dass es eine neue Memory-Map in Form einer BMM-Datei erzeugt, sobald ein Microblaze im System vorhanden ist. Das Xilinx Tool data2mem, welches zur Initialisierung und Aktualisierung von Block RAMs verwendet wird, wurde in die Toolchain integriert, um die Speicherinitialisierung für den Microblaze zu realisiern. Dieses Tool erlaubt es UCF-Dateien zu erzeugen, die in der Synthese dafür sorgen, dass die Speicher korrekt initialisiert werden. Außerdem ist data2mem in der Lage existierende Bitfiles zu aktualisieren ohne die Synthese erneut zu starten und es ist außerdem in der Lage Speicherinitialisierungen für die Simulation in Form von Verilog-Dateien zu erzeugen. Desweitern wurde im Rahmen dieser Arbeit das Konzept von Simulationslibraries in die Toolchain integriert. Dies erlaubt es VHDL-Dateien in Libraries zusammenzufassen und diese vorzukompilieren, sodass sie während der Simulation nicht erneut kompiliert werden müssen.\\
\section{Ausblick}
Im Rahmen dieser Arbeit konnte aus zeitliche Gründen allerdings nicht mehr die Integration des Microblaze GCC in die Toolchain realisiert werden. So ist die Toolchain nicht in der Lage, für den Microblaze kompatible ELF-Dateien zu erzeugen und muss daher auf extern erzeugte ELF-Dateien zurückgreifen. Desweiteren konnte keine erfolgreiche Datenübertragung über das AXI-Interface des Microblaze umgesetzt werden. In zukünftigen Arbeiten sollten daher zunächst diese beiden Probleme behoben werden. Neben diesen Punkten gibt es eine Reihe weiterer Möglichkeiten, diese Arbeit fortzusetzen. Es sollte auf jeden Fall der IP-Core für den AXI4-Bus in JConfig integriert werden, um es zu ermöglichen mehr als nur eine Peripherie an den Microblaze anzuschließen. Desweiteren wurde das optionale Ziel dieser Arbeit, die Paralellisierung eines Microblaze Programms mit \textmu\/Streams, nicht erfüllt. Hierzu wäre es notwendig, \textmu\/Streams dahingehend anzupassen, dass eine entsprechende Hardwarekonfiguration mit den neu integrierten Komponenten erstellt werden kann und die Aufrufe der Core-Konnektoren durch Aufrufe der FSL-Blöcke ersetzt werden. Außerdem wäre es interessant die verschiedenen Konfigurationsmöglichkeiten (z.b. externer Speicher mit Caches, MMU, ...) des Microblaze auszuprobieren, da im Rahmen dieser Arbeit nur die Standardkonfiguration getestet wurde. Eine Möglichkeit die Nutzbarkeit von Simulationlibraries zu verbessern, wäre es, die kompletten Bibliotheken zu erzeugen und an einem, im Fachbereich Rechnersysteme global zugänglichem Verzeichnis abzulegen. Dies ist unter Berücksichtigung der Möglichkeit, dass weitere IP-Cores von Xilinx in Zukunft integriert werden und der langen Kompilierungszeit der Libraries durchaus sinnvoll.
\begin{figure}
\centering
\includegraphics[width=0.8\linewidth, height=0.8\linewidth]{./bilder/Schaubild}
\caption{Schaubild zu den Neuerungen die im Rahmen dieser Arbeit vorgenommen wurden.}
\label{fig:Schaubild}
\end{figure}
\chapter{Zusammenfassung}
Einmal in ner halben bis ganzen Seite zusammenfassen was ich mit dieser Arbeit erreicht habe.

\section*{Abkürzungsverzeichnis}
\begin{acronym}
 \acro{CPU}{Central Processing Unit}
 \acro{MC}{Microcontroller}
 \acro{FPGA}{Field Programmable Gate Array}
 \acro{SoC}{System on Chip}
 \acro{I/O}{Input/Output}
 \acro{UART}{Universal Asynchronous Receiver Transmitter}
 \acro{IP}{Intellectual Property}
 \acro{FSL}{Fast Simplex Link}
 \acro{HDL}{Hardware Description Language}
 \acro{XML}{Extensible Markup Language)}
 \acro{UCF}{User Constraints File}
 \acro{XSD}{XML Schema Definition}
 \acro{UI}{User Interface}
 \acro{FPU}{Floating Point Unit}
 \acro{ISA}{Instruction Set Architecture}
 \acro{MMU}{Memory Management Unit}
 \acro{TLB}{Translation Lookaside Buffer}
 \acro{LMB}{Local Memory Bus}
 \acro{AXI}{Advanced eXtensible Interface}
 \acro{PLB}{Processor Local Bus}
 \acro{XCL}{Xilinx CacheLink}
 \acro{XPS}{Xilinx Platform Studio}
 \acro{SDK}{Software Development Kit}
 \acro{BSP}{Board Support Package}
 \acro{FIFO}{First In First Out}
 \acro{RAM}{Random Access Memory}
 \acro{LUT}{Look Up Table}
 \acro{BMM}{Block RAM Memory Map}
 \acro{ELF}{Executable and Linkable Format}
 \acro{VHDL}{Very High Speed Integrated Circuit Hardware Description Language}
 
 \acro{NCD}{Native Circuit Description Format}
 \acro{Sph}{SpartanMC Hex Dateiformat}
 \acro{XDL}{Xilinx Design Language Format}




\end{acronym}

\listoffigures
\listoftables
\clearpage

\makeatletter
\renewcommand\chapter{\thispagestyle{\chapterpagestyle}%
                    \global\@topnum\z@
                    \@afterindentfalse
                    \secdef\@chapter\@schapter}
\makeatother

\nocite{*}
\bibliographystyle{alphadin}
\bibliography{Literaturverzeichnis}

\end{document}