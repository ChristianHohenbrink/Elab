\chapter{Einleitung}
\section{Einführung in die Thematik}
Am Fachgebiet Rechnersysteme der TU Darmstadt und der Professsur für eingebettete Systeme der TU Dresden wurde der CPU-Core
SpartanMC entwickelt, welcher mit einer Daten- und Instruktionsbreite von 18 Bit ideal für die Verwendung in FPGAs geeignet ist.
Ebenso wurde eine Toolchain entwickelt, die es dem Nutzer ermöglicht, über die grafische Oberfläche JConfig Systeme zu beschreiben und
im Anschluss über Make-Regeln weitere Schritte wie Synthese und Simulation durchzuführen. \cite{The_SpartanMC_Project}\\
Der SpartanMC wird am Fachgebiet Rechnersysteme zur Forschung an Manycore SoC verwendet. Eine dieser Forschungsarbeiten stellt das Programm \textmu\/Streams da,
welches auf Basis des Source-to-Source Compilers Cetus \cite{CETUS} entwickelt wurde, um die Leistungsfähigkeit von Manycore Systemen zu steigern. 
Hierzu kann der Programmierer den Sourcecode mit Annotationen versehen, die den Code in verschieden Aufgabenblöcke unterteilt. Dabei ergibt sich
eine Pipeline-Struktur, in der die Ergebnisse einzelener Berechnungen über Core-Konnektoren weitergegeben werden.
\textmu\/Streams kann sowohl eine Hardwarebeschreibung für ein Manycore System generieren, als auch Software für die einzelenen Cores, 
welche mit der SpartanMC Toolchain konform sind. \cite{WeberThesis}\\
\section{Ziele der Arbeit}
Um zu zeigen, dass diese Art der Perfomancesteigerung nicht nur auf den SpartanMC begrenzt ist, ist das primäre Ziel dieser Arbeit, den Xilinx Microblaze 
in die bestehende SpartanMC Toolchain zu integerieren.\\
Hierzu soll die Konfiguration des Microblaze über den SpartanMC Systembuilder JConfig ermöglicht werden. Ebenso ist es erforderlich, für einfache
I/O-Anwendungen einen UART IP-Core für den Microblaze zu integrieren. Als Pendant zu den Core-Konnektoren soll es ebenfalls möglich sein, Fast Simplex Link (FSL) Blöcke von Xilinx
dem System hinzuzufügen, um die Kommunikation zwischen den einzelnen Prozessoren zu ermöglichen.\\
Sollte es für die Integration notwendig sein, ist die Toolchain ebenfalls anzupassen, um die Generierung von lauffähigen Microblaze Systemen zu ermöglichen.
Hierbei ist darauf zu achten, dass die Funktionalität sowohl für Systeme, die nur aus SpartanMC oder Microblaze bestehen, als auch für Systeme, die beide Prozessoren beinhalten, 
nicht negativ beeinflusst wird.\\\\
Optional wäre es wünschenswert, die erfolgreiche Paralellisierung eines Microblaze Programms durch \textmu\/Streams zu zeigen. Hierzu wäre es notwendig, 
\textmu\/Streams soweit anzupassen, dass eine entsprechende Hardwarekonfiguration mit den neu integrierten Komponenten erstellt werden kann und
die Aufrufe der Core-Konnektoren durch Aufrufe der FSL-Blöcke ersetzt werden.
\section{Gliederung der Arbeit}
Im folgenden Kapitel werden die Grundlagen für diese Arbeit beschrieben. Diese umfasst neben der Beschreibung der verwendeten Software-Tools, die Erläuterungen zum
Aufbau der SpartanMC Toolchain, sowie Erklärungen zu verwendeten Dateiformaten und Busbeschreibungen. Anschließend wird ein Konzept und die Realisierung mit allen notwendigen Schritten 
für die Integration des Microblaze in die SpartanMC Toolchain erarbeitet. In einem weiteren 
Abschnitt wird auf alle notwendigen Schritte eingegangen, um ein lauffähiges System zu generieren. Darauf folgend wird eine Evaluation zum Grad der erreichten Integration
durchgeführt. Abschließend folgt ein kurzer Ausblick auf mögliche Fortsetzungen der Arbeit, sowie eine kurze Zusammenfassung.
%\textit{kursiver Text} \\
%\texttt{besonderer Text} \\
%normaler Text
%\begin{itemize}
%\item[\textbf{1)}] \textbf{Aufzählung1}\\
%Text der Aufzählung
%\item [\textbf{2)}]\textbf{Aufzählung2}
%\item [\textbf{3)}]\textbf{Aufzählung3}
%\item [\textbf{4)}]\textbf{Aufzählung4}
%\item [\textbf{5)}]\textbf{Aufzählung5}
%\end{itemize}
%Quellenverlinkung \cite{UCS_XDL_UseCaseSenarios}

